\documentclass{article}

\begin{document}
\section{Usage}
After installation the RMextract package can be imported in a python
environment. There are five main packages: 
\begin{enumerate} 
\item EMM: This is a tiny python wrapper around some of the functionalities of
  the GeoMagnetism Library that is provided by the WMM \cite{wmm}. Can be used
  both with the WMM coefficients or the higher resolution EMM.
\item PosTools: a set of tools to read metadata from a MS, calculate line of
  sights and piercepoints through an ionospheric screen.
\item getIONEX: For downloading and combining and reading relevant IONEX
  files. Also does the interpolation of TEC values at a given piercepoint.
\item getTEC: Uses getIONEX and PosTools to get vTEC, airmass and timegrid,
  either for a given MS or for a user defined timerange, station and source
  position.
\item getRM: combines EMM, PosTools and getIONEX to extract RM values for a
  given MS or a user defined timerange, station and source
  position.
\end{enumerate}

Examples of how to use the tools can be found in the \emph{examples} directory.



\subsection{LOFAR Tools}
For usage within the LOFAR sofware environment, a script was
developed: \emph{createRMParmdb}. This script automatically creates  the
parmdb for a given Measurement Set (MS) with RotationMeasures to be used in
BBS (or NDPPP) to correct the data. The correction is calculated for the phase
center of the MS and can be  applied as a direction independent effect. Since
the main distortion to polarised data is in fact the \emph{time} variation of
the Faraday rotation as opposed to a field of view variation, this results in
a good first order correction of your polarised data. However, at the time of
writing, a direction independent rotation measure has not been implemented in
BBS or DPPP yet. Therefore, one needs to specify the name of a source (any
source) in the skymodel. And make sure that in the BBS parset the correction
is applied in the direction of the source with the same name. For long
baselines it can be useful to calculate the RM per station, this is optional
in the \emph{createRMParmdb} script. The parameter names in the parmdb are
\emph{RotationMeasure:$\langle$ stationname$\rangle$ :$\langle$ sourcename$\rangle$ }.


\subsubsection{Options}

The following list of otions is availble for the createRMParmdb script:
\begin{itemize}
\item -o,--out : Name of the output parmdb. If it exists the RM paramters will
  be added to the parmdb, unless the -r or --overwrite options is
  used. default: \emph{RMParmdb}
\item -r,--overwrite: overwrite existing parmdb.
\item -p,--patch : name of the patch or source for which to create the
  values. This will end up in the parameter name and is important for BBS when
  it looks up the parameter for correction. It needs to be the same as
  specified in the BBS parset and it should be existing in the sky
  model. Otherwise, since the correction are calculated for the phase center
  only, it really does not matter which source name is used. default:
  \emph{phase\_center}.
\item --IONprefix: this is the name of the IONEX server, e.g. CODG, ROBR,JPLG,
  etc. The IONEX files are usually named:$\langle$ IONprefix$\rangle$$\langle$
  doy$\rangle$ 0.$\langle$ year\%2000$\rangle$ I.Z. doy
  is number of the day in the year. Note
  that the data should be available on the selected ftp-server, given with
  --IONserver. Default: \emph{CODG}
\item --IONserver: name of the ftp-server for IONEX files. for CODG it is the
  default: \emph{ftp://ftp.unibe.ch/aiub/CODE/}. For ROBR use:
  \emph{ftp://gnss.oma.be/gnss/products/IONEX/}. There are other servers
  available, but the directory structure is hardcoded at the moment and should
  be the same as that of the CODE server $\langle$ year$\rangle$ . If needed, ask the developer to
  implement another directory structure for you prefered server.
\item --IONpath: location where the IONEX files are stored locally. Default: ./
\item -a,--all: Create an RM value per station. Since the RM variation over
  the array is usual minimal, the default is to only calculate for the center
  of LOFAR (postion of CS002LBA). However, when using long baselines, this option can be useful.
\item -t,--timestep: timestep in seconds. The IONEX maps have a time resolution of
  15 mintues to 2 hours. Although interpolation is applied, it is usually not
  necessary to use a time resolution smaller than 15 minutes. Caclulating the
  corrections at full resolution can be time consuming.  Default: 900 seconds
\item -e,--smart\_interpol': Float parameter describing how much of the Earth
  rotation is taken in to account in interpolation of the IONEX files. 1.0
  means time interpolation assumes ionosphere rotates in opposite direction of
  the Earth. 0.0 (default) means no rotation applied. See \cite{ionex} for
  description of this interpolation method.

\end{itemize}
 


\begin{thebibliography}{}

\bibitem[{\textit{Chulliat et al}(2014)}]{wmm} Chulliat A., Macmillan S., Alken P.,
  Beggan C., Nair M., Hamilton B., Woods A., Ridley V., Maus S., Thomson A. (2014), \textit{The US/UK World Magnetic Model for 2015-2020, NOAA
  National Geophysical Data Center, Boulder, CO}, doi: 10.7289/V5TH8JNW.
\bibitem[{\textit{Schaer et al}(1997)}]{ionex} Schaer, S., W. Gurtner, and J. Feltens (1997), \textit{IONEX: The IONosphere Map
        EXchange Format Version 1}, February 25, 1998, in \textit{Proceedings of
        the 1998 IGS Analysis Centers Workshop, ESOC, Darmstadt, Germany,
        February 9-11, 1998}.
\end{thebibliography}
\end{document}
